\documentclass{article}
\usepackage{pdfpages}
\usepackage{graphicx}  % For PNG
\usepackage[left=2cm, right=2cm, top=2cm]{geometry}
\usepackage{minted}
% Give Table of Contents Hyperlinks
\usepackage{hyperref}
\hypersetup{
    colorlinks,
    citecolor=black,
    filecolor=black,
    linkcolor=black,
    urlcolor=blue
}
\pagenumbering{gobble}
% \pagenumbering{roman} % set the numbering style to lowercase letter

\title{\textbf{Homework 4}}

\author{MacMillan, Kyle}
\date{October 26, 2018}

\begin{document}


\addcontentsline{toc}{section}{Title}
\maketitle

\newpage
\tableofcontents
\addcontentsline{toc}{section}{Table of Contents}
\newpage
\listoffigures
\addcontentsline{toc}{section}{List of Figures}

\pagenumbering{roman}   % Set TOC page numbering to lowercase roman numerals



%%%%%%%%%%%%%%%%%%%%%%%%%%%% INTRO SECTION %%%%%%%%%%%%%%%%%%%%%%%%%%%%
\newpage
\hypersetup{
    colorlinks,
    citecolor=blue,
    filecolor=black,
    linkcolor=blue,
    urlcolor=blue
}
\pagenumbering{arabic}  % Set content page numbering to arabic numerals

\setcounter{page}{1}
\newpage
\section{\textbf{Problem 5.2}}
The meat and potatoes of this problem is in bump and gps callbacks, the 
rest is filler but can be found \href{https://github.com/macattackftw/RoboticsHW/blob/master/HW4/Problem2.py}{here}. The environment used is shown in Figure \ref{fig:stuck1}.

\begin{minted}{python}
def bumpCallback(self, msg):
    hit_obj = False
    for i in range(len(msg.data)):
        hit = unpack('b', msg.data[i])[0]
        if hit != 0:
            hit_obj = True

    if hit_obj:
        # Turn right
        self.bumps += 1
        self.setVel(0.0, 2.0)
    else:
        self.bumps = 0
        self.setVel(2.0, 2.0)

def gpsCallback(self, msg):
    # we need to move to goal if we are not bumping a wall
    if self.bumps == 0:
        # Wraps deals with the robot if it spins around somehow
        wraps = np.abs(int(msg.theta / (2 * np.pi)))
        theta = np.fabs(msg.theta) - (wraps * 2 * np.pi)
        beta = np.arctan2(self.goal[1] -
                          msg.y, self.goal[0] -
                          msg.x)
        k = 0.25
        alpha = beta - theta
        w1 = 2.0 + k * alpha
        w2 = 2.0 - k * alpha
        self.setVel(w1, w2)

\end{minted}

\begin{figure}[h]
    \centering
    \includegraphics[pages=1]{stuck1}
    \caption{Problem 5.2 Stuck Robot}
    \label{fig:stuck1}
\end{figure}

\newpage
\section{\textbf{Problem 9}}
\subsection{Problem 9.1}
Figure \ref{fig:9.1} shows the required plot. The robot location is:\\
$x = 803.84497$\\
$y = 485.52026$\\
$z = 517.26977$\\

With an error of $E = 2720.65$


\begin{figure}[h]
    \centering
    \includegraphics[pages=1]{problem9-1}
    \caption{Problem 9.1}
    \label{fig:9.1}
\end{figure}

\subsection{Problem 9.2}
\noindent $\lambda = c * 10 MHz$\\
$\lambda = 30\ meters$\\


Assuming phase shift $\theta = 10$ we can plug that into our formula to get

$$D' = L + \frac{\theta}{2\pi}\lambda$$
Therefore $D = \frac{D'}{2} = 0.833333333 + 15k$ where $k$ denotes an integer 
interval. We make the assumption that L is arbitrarily small compared to the 
distance travel and is therefore set to $0$. If the system has noise we will 
have to identify a range for $\frac{D'}{2}$, in this case it's $0.825\ to\ 
0.841666667 + 15k$. In order to differentiate between 20 and 250 meters we would 
need a second system at a $\lambda$ multiple that doesn't overlap before a 
distance of 250 meters.

\section{\textbf{Problem 10}}
\subsection{Problem 10.1}
\noindent$f = 0.8cm$\\
$b = 30cm$\\
$a = tan^-1 \big(\frac{z}{b-x}\big)$\\
$u = \frac{fx}{z}$\\

\noindent Given the above formulas we can say $a$ is in the range of: 
$45 < a < 90$ and for $u$: $3 < u < 45$.

\subsection{Probelm 10.2}
\noindent$e = 10\%$\\
$v_1 = 0.2cm$\\
$v_2 = 0.3cm$\\
$z = \frac{fb}{v_1 + v_2}$\\

\noindent Given the above formulas we can say that $f * b = 0.7 * 10 = 7$ but the range 
of z is dependent on $v_1 + v_2$, or:\\
$\frac{7}{0.18 + 0.27} <= z <= \frac{7}{0.22 + 0.33}$\\

\noindent With zero error we would expect $z = 14$, on the low end we expect $z = 12.72$ 
and on the upper end we expect $z = 15.56$, leaving an error of $3.7037\%$.
\end{document}
